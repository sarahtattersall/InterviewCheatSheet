\subsection{Graph}
\begin{center}
\begin{longtable}{|p{4cm}|p{4cm}|p{7cm}|}
\hline
\textbf{Name} & \textbf{Worst Complexity} & \textbf{Description} \\
\hline
Depth First Search & O(\textbar E\textbar + \textbar V\textbar )
           & \begin{enumerate}
                \item Push first vertex onto a stack
                \item Pop vertex off the stack \& process
                \item Loop through vertex successors, if not already seen
                      add it to the stack
                \item Repeat until stack is empty
              \end{enumerate}\\
\hline

Breadth First Search & O(\textbar E\textbar + \textbar V\textbar )
           & \begin{enumerate}
                \item Push first vertex onto a queue
                \item Pop vertex off the queue \& process
                \item Loop through vertex successors, if not already seen
                      add it to the stack
                \item Repeat until stack is empty
              \end{enumerate}\\
\hline

BST pre-order & O(n) 
              & {\begin{lstlisting}[language=Java]
Deque<TreeNode> stack = new ArrayDeque<>();
stack.push(root);
while(!stack.isEmpty()) {
  TreeNode node = stack.pop();
  // Process node here
  if (node.right != null) {
    stack.push(node.right);
  }
  if (node.left != null) {
    stack.push(node.left);
  }
}
                  \end{lstlisting}

                } \\
\hline


BST in-order & O(n) 
              & {\begin{lstlisting}[language=Java]
Deque<TreeNode> stack = new ArrayDeque<>();
TreeNode node = root;
while (!stack.isEmpty() || node != null) {
  if (node != null) {
    stack.push(node);
    node = node.left;
  } else {
    node = stack.pop();
    // Process node here
    node = node.right;
  }
}
                  \end{lstlisting}

                } \\
\hline

BST post-order & O(n) 
              & {\begin{lstlisting}[language=Java]
Deque<TreeNode> stack = new ArrayDeque<>();
stack.push(root);
TreeNode prev = null;
while(!stack.isEmpty()) {
  TreeNode current = stack.peek();
  if (prev == null || prev.left.equals(current) || prev.right.equals(current)) {
    if (current.left != null) {
      stack.push(current.left);
    }
    else if (current.right != null) {
      stack.push(current.right);
    }
  } else if (current.left.equals(prev)) {
    if (current.right != null) {
      stack.push(current.right);
    }
  } else {
    //Process current here
    stack.pop();
  }
  prev = current;    
}
                  \end{lstlisting}

                } \\
\hline

 
Prims MST priority queue/min heap & O(\textbar V\textbar log\textbar V\textbar + \textbar E\textbar log\textbar V\textbar ) $\implies$ O(\textbar E\textbar log\textbar V\textbar) 
            & \begin{enumerate}
                \item Push first vertex's edges onto a priority queue
                \item While the priority queue is not empty do:
                \item Poll the smallest edge and add it to the mst, if its
                      connecting vertex has other edges to it in the 
                      priority queue, remove these because this edge was smaller
                      and they don't need looking at.
              \end{enumerate}\\

\hline

Kruskals MST & O(\textbar E\textbar log\textbar V\textbar) 
         & \begin{enumerate}
                \item Push all edges onto a priority queue
                \item Pop first V edges off the queue, for each edge:
                \item if egde is not already in a connected tree of the answer 
                      add it to the mst. This is tested via a union find data
                      structure

              \end{enumerate}\\

\hline

Dijkstra & O((\textbar V \textbar + \textbar E \textbar) log \textbar V \textbar
         & \begin{enumerate}
                \item Assign distance to the initial node 0 and the others 
                      $\infty$
                 \item Add initial node to a priority queue and then whilst the
                       queue is not empty do:
                 \item Pop closest vertex off queue, for all its edges if any
                       edges make the distance closer than the current recorded distance update
                       the vertex in the priority queue
                 \item Final path can be used via back tracking
              \end{enumerate}\\

\hline

A* & 
   & \begin{enumerate}
        \item A* uses a best first search, keeping a sorted priority queue
              of alternative path segments along the way
        \item It uses a heuristic $f(x)$ made up of $g(x)$ which is the 
              past path cost, and $h(x)$ which is the future path cost estimation
        \item Starting with the initial node it maintains a priority queue of nodes
              (known as the fringe). The lower $f(x)$ the higher the priority
      \end{enumerate}\\ 
\hline


\end{longtable}
\end{center}
